% "Лабораторная работа № 7"

\documentclass[a4paper,12pt]{article} % тип документа

% report, book

% Русский язык

\usepackage[T2A]{fontenc}			% кодировка
\usepackage[utf8]{inputenc}			% кодировка исходного текста
\usepackage[english,russian]{babel}	% локализация и переносы
\usepackage{amsmath}

% Математика
\usepackage{amsmath,amsfonts,amssymb,amsthm,mathtools} 


\usepackage{wasysym}

\usepackage{hyperref}

%Заговолок
\author{Афанасьев А.Д. 1 гр. 1 подгр.}
\title{Особенности технологии создания текста с формулами в \LaTeX{}}
\date{07.12.2021}


\begin{document} % начало документа
\maketitle
\newpage
\section{Задания}
\subsection{Задание 1}
\begin{enumerate} 
  \item 
  На рисунке дана функция. Коэффициенты $a, b, c$ являются константами, а $x$ находится в интервале $[-10; 18]$ и изменяется с шагом $h$, значение которого вводится с клавиатуры. Найдите все значения функции для заданных $x$. $$y = ax^{2}+bx+c$$
  \item
  На рисунке дана функция. Найти значение переменной $n$, при котором значение функции превысит $1000$. $$y = 2^{n-1}+3$$
  \item
  На рисунке дана функция. В данной функции $t, a, s - const, x$ - вводится с клавиатуры. Найдите значение функции.
    \begin{equation*}
        y = 
        \begin{cases}
            t, &\text{при $x\geq 3$}\\
            ax-s, &\text{при $\in (-5.5;3)$}\\
            x^{3}, &\text{при $x\leq -5.5$}
        \end{cases}
    \end{equation*}
\end{enumerate}
\subsection{Задание 2}
\begin{enumerate} 
    \item
    Вычислить значения функции $y(x)$ для каждого $x$. Коэффициенты $t, k, s$ являются константами и ввиодятся с клавиатуры. Значение $x$ находится в интервале $[-25; 15]$ и изменяется с шагом $1$.
    $$y = t\cdot x^{3} + k\cdot x + s$$
    \item
    Изменяя значение переменной $k$ (начальное значение $k=1$, шаг 1), найдите при каком $k$ значение функции $y(k)$ превысит $1200$.
    $$y = 2^{k+2}-5$$
    \item
    В данной функции $w, n, c$ - константы, $x$ - вводится с клавиатуры. Найти значение функции.
    \begin{equation*}
        y = 
        \begin{cases}
            w^{2}, &\text{при $x\geq 1.5$}\\
            n\cdot x+9, &\text{при $\in (-12;1.5)$}\\
            c-x, &\text{при $x\leq -12$}
        \end{cases}
    \end{equation*}
\end{enumerate}
\subsection{Задание 3}
    \large$$\int\frac{dx}{\ln x}=\ln\left | \ln x \right|+\sum_{i=1}^{\infty}\frac{(\ln x)^{i}}{i \cdot i!}$$
    
    $$\int \frac{dx}{(\ln x)^{n}}=-\frac{x}{(n-1)(\ln x)^{n-1}}+\frac{1}{n-1} \int\frac{dx}{(\ln x)^{n-1}}\text{ для } n\ne1$$ 
    
    $$\int x^{m}\ln x dx = x^{m+1}\begin{pmatrix}
        \frac{\ln x}{m+1}-\frac{1}{(m+1)^{2}}
    \end{pmatrix}\text{ для } m\ne-1$$
   
    $$\int x^{m}(\ln x)^{n} dx = \frac{x^{m+1}(\ln x)^n}{m+1}-\frac{n}{(m+1)^{2}}\int x^{m}(\ln x)^{n-1} dx\text{ для } m\ne-1$$
    
    $$\int \frac{(\ln x)^n dx}{x}=\frac{(\ln x)^{n+1}}{n+1}\text{ для } n\ne1$$
    
    $$\int \frac{\ln x\;dx}{x^m}=-\frac{\ln x}{(m-1)x^{m-1}}-\frac{1}{(m-1)^2x^{m-1}}\text{ для } m\ne1$$
    
    $$\int \frac{(\ln x)^n\;dx}{x^m}=-\frac{(\ln x)^n}{(m-1)x^{m-1}}-\frac{n}{m-1}\int \frac{(\ln x)^{n-1}dx}{x^m}\text{ для } m\ne1$$
\end{document}